\chapter{Conclusions}
We saw that the existence of negative aboslute temperatures is a direct result of the formal definition of temperature we use in thermodynamics. \\
In the first chapter I showed why this definition makes sense according to the general laws of thermodynamics which match our empirical observations. In particular I showed that there is a class of 
functions that assume the same value for systems that are in equilibrium with each other, and different values for systems that are not at reciprocal equilibrium. Any of these functions is a temperature. \\
By a combinatorial approach we saw that systems at equilibrium share the same value of the function $\frac{\partial S_i}{\partial E_i}$ which can then be considered a temperature. \\
The two levels system was then introduced showing some of the charateristic behaviours of the negative absolute temperatures as that they are hotter than any positive ones and that they describe systems with an inverted population state. \\
A general set of criteria to estabilish whether a system admits negative temperatures or not was then introduced, namely the Ramsey's criteria. \\
After that I discussed the first experimental observation of negative temperatures carried by Purcell and Pound, who observed a LiF spin lattice in an inverted population state by rapidly inverting a strong magnetic field. \\
Since the definition of temperature is a direct consequence of the definition of entropy, chapter \ref{ch:entropy} was devoted to the discussion of the historical issues and debates on the choice of the correct definition. Boltzmann's entropy was initially thought to be 
missing a thermodynamic condition, but was showed later to match it in the limit of a high number of degrees of freedom. Gibbs' entropy was instead showed to fail a basic condition in a specific example, that is to assume the same value for systems at equilibrium, and different values for systems out of equilibrium. \\
The last chapter was devoted to two simulations. The numerical analysis of the Ising model showed that the behaviour of a ferromagnetic system at negative temperatures is the same of an antiferromagnetic system at positive temperatures and viceversa. Instead the numerical experiment in which two system were put into contact, one of which at negative temperature,
showed that the system at negative temperature released heat to the one at positive temperature. \par 
One may wonder if negative temperatures have a direct application in everyday life: this is indeed the case. For example it was proved that Carnot machines working at negative temperature can obtain an efficiency equal to unity, 
giving an intuition on how more complex machines could produce more work when allowed to work at negative temperature. \\
Probably the most relevant application of negative temperatures is in the field of cosmology: negative absolute temperatures were found to be linked to negative pressures, fundamental to describe the accelerating expansion of the universe.