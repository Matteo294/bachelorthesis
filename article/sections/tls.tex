The most simple system that can exhibit negative temperatures is the two levels system (TLS). \\
A TLS is a system (for example a particle) for which only two values of energy are admitted, say $E_1$ and $E_2$. Let us denote the 
corresponding eigenstates by $\ket{1}$ and $\ket{2}$. \\
Let us now consider a system composed of $N$ TLS. It is convenient to introduce the occupation numbers $n_1, n_2$ which denote,
respectively, the number of TLS at energy $E_1$ and $E_2$. If we set $E_1=\epsilon$ and $E_2=0$ for simplicity, the energy of the system is
\begin{equation}
    E = n_1 E_1 + n_2 E_2 = n_1\epsilon
    \label{eq:TLS_ensemble_energy}
\end{equation}
where $n_1 + n_2 = N$. \\
One macrostate of the system is thus identifies by its energy and the total number of particles. The number of microstates corresponding 
to one given microstate is the number of ways in which one can rearrange the particles in a way such that the total energy remains fixed, that is 
\begin{equation*}
    \Omega(E, N) = \frac{N!}{N_1!N_2!} = \frac{N!}{N_1! \, (N-1)!}
\end{equation*}
which corresponds to the Boltzmann entropy 
\begin{equation*}
    S(E, N) = k_B\ln\left(\frac{N!}{N_1! \, (N-1)!}\right)
\end{equation*}
In the limit of large $N$ the last expression can be expanded using using Stirling's formula $\ln(N!) \approx N\ln N$ which yields 
\begin{equation*}
    S(E, N) \approx N \ln \left(\frac{N}{N-n_1}\right) + n_1 \ln\left(\frac{N-n_1}{n_1}\right)
\end{equation*}
By using relation \ref{eq:TLS_ensemble_energy}
\begin{gather*}
    \frac{1}{T} = \frac{\partial S}{\partial E} = \frac{\partial S}{\partial n_1} \, \frac{\partial n_1}{\partial E} =
    \frac{k_B}{\epsilon} \, \ln\left(\frac{N - n_1}{n_1}\right) = -\frac{k_B}{\epsilon} \, \ln\left(\frac{E}{N\epsilon - E}\right)
\end{gather*}
where in the last step I used equation \ref{eq:TLS_ensemble_energy} again. \\
