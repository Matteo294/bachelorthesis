\subsubsection*{Two-levels systems admit negative temperatures}
The most simple system that can exhibit negative temperatures is the two levels system (TLS). \\
A TLS is a system (for example a particle) for which only two values of energy are admitted, say $E_1$ and $E_2$. Let us denote the 
corresponding eigenstates by $\ket{1}$ and $\ket{2}$. \\
Let us now consider a system composed of $N$ TLS. It is convenient to introduce the occupation numbers $n_1, n_2$ which denote,
respectively, the number of TLS at energy $E_1$ and $E_2$. If we set $E_1=\epsilon$ and $E_2=0$ for simplicity, the energy of the system is
\begin{equation}
    E = n_1 E_1 + n_2 E_2 = n_1\epsilon
    \label{eq:TLS_ensemble_energy}
\end{equation}
where $n_1 + n_2 = N$. \\
One macrostate of the system is thus identified by its energy and the total number of particles. The number of microstates corresponding 
to one given microstate is the number of ways in which one can rearrange the particles in a way such that the total energy remains fixed, that is 
\begin{equation*}
    \Omega(E, N) = \frac{N!}{N_1!N_2!} = \frac{N!}{N_1! \, (N-1)!}
\end{equation*}
which corresponds to the Boltzmann entropy 
\begin{equation}
    S(E, N) = k_B\ln\left(\frac{N!}{N_1! \, (N-1)!}\right)
    \label{eq:TLS_entropy_N}
\end{equation}
In the limit of large $N$ the last expression can be expanded using using Stirling's formula $\ln(N!) \approx N\ln N$ which yields 
\begin{equation*}
    S(E, N) \approx N \ln \left(\frac{N}{N-n_1}\right) + n_1 \ln\left(\frac{N-n_1}{n_1}\right)
\end{equation*}
By using relation \ref{eq:TLS_ensemble_energy}
\begin{gather*}
    \frac{1}{T} = \frac{\partial S}{\partial E} = \frac{\partial S}{\partial n_1} \, \frac{\partial n_1}{\partial E} =
    \frac{k_B}{\epsilon} \, \ln\left(\frac{N - n_1}{n_1}\right) = -\frac{k_B}{\epsilon} \, \ln\left(\frac{E}{N\epsilon - E}\right)
\end{gather*}
where in the last step I used equation \ref{eq:TLS_ensemble_energy} again. \\
A plot of the temperature as a function of the system's energy is reported in figure \ref{fig:temperature_TLS}. Negative temperatures occure in the region in which $E > \frac{N\epsilon}{2}$, which correspond to the states in which there are more particles 
in the excited state than in the lower one.
\begin{figure}
    \centering 
    \includegraphics[scale=0.5]{images/temperature_TLS.png}
    \caption{The plot reports the temperature as a function of the energy in a two-levels system. When there are more excited particles than those in the lower state the system exhibits negative absolute temperatures.}
    \label{fig:temperature_TLS}
\end{figure}
Let us recall what we mentioned at the end of \hyperref{sec:temperature}{section 2}: a system whose maximum energy state is allowed by only one or few microstates may exhibit a decreasing entropy as a function of the energy, hence admitting negative temperatures. This is 
exactly the case of a TLS for which the maximum energy state corresponds to exactly one precise microstate, that is when all the particles are in the excited state. This of course corresponds to a null entropy. Analogously, the same happens at the minimum energy for which there is only 
one corresponding microstate and the entropy is null. For all the other states the entropy is non-zero and is given by formula \ref{eq:TLS_entropy_N}. The whole expressions as a function of the energy can be easilly obtained by \ref{eq:TLS_entropy_N} by multiplying and diving by $\epsilon$ both inside and outside the logarithm
\begin{equation*}
    S(E, N) = N\ln\left(\frac{N\epsilon}{N\epsilon - E}\right) + \frac{E}{\epsilon} \ln\left(\frac{N\epsilon - E}{\epsilon}\right)
\end{equation*}
and is reported in figure \ref{fig:TLS_entropy_E}. \\
\begin{figure}
    \centering 
    \includegraphics[scale=0.5]{images/temperature_TLS.png}
    \caption{}
    \label{fig:TLS_entropy_E}
\end{figure}
Let us now formalize this insight 
\subsubsection*{Ramsey's criteria}
Ramsey %\cite{Ramsey}}
provided 3 conditions under which a thermodynamic system admits negative temperature
\begin{enumerate}
    \item 
\end{enumerate}