Everyone has an intuitive idea of what temperature is in everyday's life, for example as a quantity that provides us information 
about the environment. \\ 
Temperature is measured everyday via thermometers, particular devices that provide a number that quantifies the grade of \emph{hotness} of what we are observing. \\
An adequate measure of temperature is given by a number and a physical unit. All the thermometers give the same number when used to measure the temperature of the same object, provided that 
the number is expressed always in the same physical unit. The most common temperature unit is the degree Celsius: this scale of values is normally estabilished by fixing the values of temperature of two well known
physical phenomena and assuming a linear relation between temperature and the physical property of the material used as a thermometer, may it be the height of a liquid column in a bulb or an elastic deformation of a certain solid material.
A roughly good calibration of a thermometer in the Celsius scale can be done by assigning a thermometer a value of $0^\circ$C) (0 degrees Celsius) at the frozen point of water, and $100^\circ$C at the boiling point of water. A more sophisticated and precise way to calibrate a thermometer 
on the Celsius scale consists in assigning the value of $-273.15^\circ$C to the \emph{coldest} temperature admitted according to nowadays' physics, and $0.01^\circ$ C to the triple point of whater, that 
is the precise situation in which water coexists in liquid, solid and gaseous form. \\
Another interesting temperature scale is the one measured in degrees Kelvin (K), called the \emph{absolute temperature}. This scale of temperature is 
defined in a way such that it gives a values of $0$ K at the coldest possible situation admitted by physics, the so called \emph{absolute zero} point: this means that no physical systems can be cooled more than a system 
whose absolute temperature is $0$ K. \\
The purpose of this thesis is to introduce the concept of \emph{negative absolute temperatures}. This does not constitutes a contradiction to what told before because negative absolute temperatures should not be searched below the absolute zero, but rather 
below infinity: negative absolute temperatures are hotter that all the positive temperatures. The world of negative absolute temperatures is often accompanied by strange phenomena: for example, in this range of temperatures,
a cooler body get cooled down spontaneously giving heat to a hotter body, increasing the temperature of the latter. \par
\vspace{10pt}
In the first chapter of this thesis I will first introduce the concept of temperature in a rigorous way, according to physical laws, justifying the existence of negative absolute temperatures basing on thermodynamical arguments. \\
A simple system that admits negative temperatures will then be presented in chapter \ref{sec:TLS}, namely the two level system. \\
In chapter \ref{sec:PandP} I will then show that negative absolute temperatures were experimentally observed by Purcell and Pound in 1950. 
After that discovery, some criticisms were moved against the definition of the entropy used the define the absolute scale, putting in doubt the existence of negative temperatures: this will be discussed in chapter \ref{sec:entropy}. \\
Finally a simulation of a system at negative temperature will be presented in the last chapter.